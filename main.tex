\documentclass[12pt]{article}


\usepackage[brazilian]{babel}
\usepackage[utf8x]{inputenc}
\usepackage{amsmath}
\usepackage{graphicx}
\usepackage[colorinlistoftodos]{todonotes}
\usepackage{geometry}
\usepackage{verbatim}
\usepackage{microtype}
\usepackage[numbers]{natbib}
\usepackage[hidelinks]{hyperref}
\usepackage{amsfonts}
\usepackage{color}
\geometry{a4paper}

\title{Título}
\author{Cezar Augusto Contini Bernardi \\ \emph{Universidade Federal de Santa Maria}}

\begin{document}
\maketitle


\section{Identificação}

\begin{description} \itemsep 0pt

\item{\textbf{Resumo:}} RESUMO


\item{\textbf{Período de Execução}}: Março de 2016 até Julho de 2016

\item{\textbf{Unidades Participantes}}: 

    Curso de Ciência da Computação
    
    Laboratório de Sistemas de Computação
    
    
\item{\textbf{Área do Conhecimento}}: Ciência da Computação

\item{\textbf{Linha de Pesquisa}}: ?***********************?

\item{\textbf{Titulo do Projeto}}: Trabalho de Conclusão de Curso

\item{\textbf{Participantes}}:
\\Prof João Vicente Ferreira Lima - Orientador
\\Cezar Augusto Contini Bernardi - Orientando 
\end{description}


\section{Introdução}

Jogando umas palavaras chave aqui.

Orquestração/automatização, Amazon Web Services, programação paralela, recursos computacionais, Cloud Formation/puppet/scripting, educação, 

Este projeto visa o uso de [ferramenta de orquestração] aliado ao Amazon Web Services (AWS) para proporcionar um ambiente prático e funcional no ensido de programação paralela na Universidade Federal de Santa Maria. AWS é uma plataforma ideal para tal cenário pois dispõe diversos recursos interessantes à aprendizagem de paralelismo.

Dentre esses recursos pode-se listar a formação de clusters, multithreading e GPGPU, possibilitando aprendizagem de paralelismo local e distribuído, em mais de uma forma.


\section{Objetivos}
\subsection{Objetivo Geral}
Automatizar a implantação 

\subsection{Objetivos Específicos}
\begin{itemize}
	\item ...
	\item ...
	\item ...
	\item ...
    
\end{itemize}

\section{Justificativa}
Justificativa

Justificativa

\section{Revisão de Literatura}

Revisão

\section{Metodologia}
Metodologia

\section{Plano de Atividades e Cronograma}
\begin{enumerate}
\item \label{activity:cloud} \textbf{Estudo de ...}
...
\item \label{activity:develop} \textbf{Desenvolvimento...}
...
\item \label{activity:exec} \textbf{Teste/experimentação}
...
\item  \label{activity:updates} \textbf{Atualização de estados dos experimentos}
...
\end{enumerate}
\textbf{Cronograma}

\begin{table}[ht]
\centering
\begin{tabular}{c|ccccc}
	Etapa & Março & Abril & Maio & Junho & Julho \\ \hline
	\ref{activity:cloud} & \checkmark & & & \\
	\ref{activity:develop} & \checkmark & \checkmark & & \\
	\ref{activity:exec} & & \checkmark & \checkmark & \checkmark & \\
	\ref{activity:updates} & & &\checkmark & \checkmark & \checkmark \\
\end{tabular}
\caption{Cronograma de Atividades}

\end{table}
\section{Recursos}
Recursos
\section{Resultados Esperados}
Resultados esperados
\bibliographystyle{abbrvnat}
\bibliography{../graphics,../languages}

\end{document}