\documentclass[12pt]{article}


\usepackage[brazilian]{babel}
\usepackage[utf8x]{inputenc}
\usepackage{amsmath}
\usepackage{graphicx}
\usepackage[colorinlistoftodos]{todonotes}
\usepackage{geometry}
\usepackage{verbatim}
\usepackage{microtype}
\usepackage[numbers]{natbib}
\usepackage[hidelinks]{hyperref}
\usepackage{amsfonts}
\usepackage{color}
\geometry{a4paper}

\title{Automatização do Gerenciamento e Implantação de Instâncias AWS EC2 para Ensino de Computação Distribuída}
\author{Cezar Augusto Contini Bernardi \\ \emph{Universidade Federal de Santa Maria}}

\begin{document}
\maketitle


\section{Identificação}

\begin{description} \itemsep 0pt

\item{\textbf{Resumo:}} Este projeto busca a melhora do processo de ensino aprendizagem de computação distribuída utilizando cloud computing. Infelizmente, o uso dessas ferramentas é laborioso, principalmente quando se aplica a uma turma completa de Programação Paralela ou Sistemas Distribuídos, por exemplo. Portanto, o foco deste trabalho é a automatização do gerenciamento de IaaS da Amazon Web Services, visando facilitar a implantação e uso desses recursos para ensino.


\item{\textbf{Período de Execução}}: Março de 2016 até Julho de 2016

\item{\textbf{Unidades Participantes}}: 

    Curso de Ciência da Computação
    
    Laboratório de Sistemas de Computação
    
\item{\textbf{Área do Conhecimento}}: Ciência da Computação

\item{\textbf{Linha de Pesquisa}}:  Processamento Paralelo e Distribuído

\item{\textbf{Titulo do Projeto}}: Trabalho de Conclusão de Curso

\item{\textbf{Participantes}}:
\\Prof João Vicente Ferreira Lima - Orientador
\\Cezar Augusto Contini Bernardi - Orientando 
\end{description}


\section{Introdução}

Os serviços na nuvem, também conhecidos como cloud computing, tem expandido muito ultimamente, proporcionando diversas oportunidades de aplicação, nem sempre de fácil uso. Esses serviços podem ser classificados em três principais categorias, sendo elas Infraestrutura como um Serviço (IaaS), Plataforma como um Serviço (PaaS) e Software como um Serviço (SaaS) \cite{xaas}.

O modelo PaaS oferece abstração de diversas camadas, como sistema operacional, rede e infraestrutura, deixando-as prontas para o uso no desenvolvimento de aplicativos a serem disponibilizados na nuvem. Com isso o desenvolvedor pode focar mais no desenvolvimento. Um exemplo notável de tal serviço é o Google App Engine.

De forma semelhante, o SaaS disponibiliza aplicativos prontos para uso, novamente citando o Google como exemplo: Google Apps For Work (Drive, Gmail, Docs, entre outros).

Porém, este projeto visa o uso de IaaS, que foca em dar liberdade ao usuário montar a sua própria infraestrutura, gerenciando suas próprias máquinas virtuais, rede, sistema operacional e ambiente de desenvolvimento. Tal formato é interessante pois ele possibilita a customização da infraestrutura para cada caso desejado.

A plataforma de escolha para realização da proposta é a Amazon Web Services (AWS). Tal plataforma foi a escolha pois ela provê acesso gratuito no período de 1 ano, com limitação de poder computacional, mas o  suficiente para realização do projeto. Outro ponto interessante é que a AWS tem um programa chamado AWS Educate \cite{awsedu}, que tem por objetivo prover crédito monetário para uso do AWS em educação.

Outra razão para essa escolha se dá pelo fato de toda a infraestrutura física ser gerenciada pela Amazon. Todos esses aspectos somados apresentam uma grande vantagem em relação ao uso de OpenStack, outra plataforma semelhante a AWS. Uma vantagem do OpenStack é que poderia ser implementado em servidores próprios, caso estivessem disponíveis.

Mais especificamente, visa-se o uso do AWS Elastic Compute Cloud (EC2). Esse tipo de instância é focada em processamento, oferecendo mais controle dos aspectos computacionais a serem utilizados \cite{ec2}.

Os valores associados ao uso do AWS EC2, com instâncias otimizadas para computação localizados no leste dos Estados Unidos da America variam de 0,105 à 1,68 USD por hora. Por exemplo, o custo de utilização de 4 instâncias intermediárias, com 8 vCPUs, 15 GB de memória RAM e 160 GB de armazenamento em SSD por um dia inteiro custaria 40,32 USD.

Quanto ao ensino-aprendizagem, as melhoras seriam notáveis, pois experimentos com MPI para programação paralela ou RMI/RPC para programação distribuída seriam realizados em ambientes realistas. Um ponto importante é que cada aluno teria acesso a esse ambiente para si só e poderia experimentar mais e compreender melhor o funcionamentos dessas técnicas de programação.

Porém, a implantação desse modelo é uma tarefa minusciosa e repetitiva, tornando mais difícil para professores fazerem uso dela. Este projeto visa automatizar tal tarefa, torná-la trivial, assim como tornar o uso desses serviços na nuvem mais atrativo do ponto de vista acadêmico.

De forma mais específica, a orquestração desse ambiente pode ser feita de mais de uma forma, sendo elas, por exemplo, o uso de scripts puros, de módulos Puppet ou Amazon Cloud Formation. Tais modelos serão estudados e o mais adequado será escolhido para implementação.

O uso de scripts puros significa ter de implementar manualmente funções que se comunicam com o AWS e realizam as operações desejadas, sendo que cada configuração das instâncias requer uma operação. Esse modelo pode tornar-se confuso e trabalhoso quando se trata de diversas instâncias.

Já o Puppet conta com a desvantagem de requerer uma instância rodando sem parar como nodo master, onde os módulos com a definição da infraestrutura estão configurados, e de onde cada nó agente leria as informações e se configuraria. A vantagem é que neste modelo as configurações são bastante flexíveis, possibilitando modificações e personalizações com grande facilidade \cite{puppetaws}.

O Amazon Cloud Formation é possivelmente o modelo mais consistente, pois ele opera a partir de um arquivo JSON descrevendo toda a infraestrutura, desde quantidade de instâncias até os recursos instalados em cada uma delas \cite{awscf}.


\section{Objetivos}
\subsection{Objetivo Geral}
Criar um modelo para automatizar a implantação de um serviço de infraestrutura em cloud para o ensino de computação distribuída e paralela.

\subsection{Objetivos Específicos}
\begin{itemize}
	\item Estudar Amazon Web Services e ferramentas de orquestração para tal.
	\item Implementar a orquestração.
	\item Tornar essa implementação disponível publicamente para uso.
    
\end{itemize}

\section{Justificativa}
Este trabalho tem grande potencial para melhorar o processo de ensino aprendizagem de programação paralela e distribuída, facilitando o acesso à recursos já disponíveis, porém restritivos. A automatização de implantação de tais recursos é um grande passo para a adoção mais ampla e fácil.


\section{Revisão de Literatura}

\subsection{Trabalhos Relacionados}
Nesta seção serão apresentados trabalhos com caraterísticas semelhantes as desejadas no desenvolvimento deste projeto.

\subsubsection{Parallel Computing at Stanford University}
Para o ensino de programação paralela na Universidade de Stanford, o professor Ph.D. Elliott Slaughter desenvolveu um script em Python para automatizar o gerenciamento da infraestrutura AWS utilizada \cite{stanford}.

A infraestrutura a ser implantada pode ser configurada através de um arquivo JSON, para modificações mais gerais, e do script para modificações mais internas aos nodos.

\section{Metodologia}
Serão analisadas e estudadas ferramentas de orquestração para Amazon Web Services, com o intuito de escolher a mais adequada.

Tendo definido a ferramenta, será realizado o desenvolvimento, procurando a melhor solução para facilitar o processo de integração da ferramenta na Universidade Federal de Santa Maria, assim como a melhor infraestrutura para ensino dos conteúdos aplicáveis.

\section{Plano de Atividades e Cronograma}
\begin{enumerate}
\item \label{activity:cloud} \textbf{Estudo do Amazon Web Services: }
Compreender o funcionamento e as técnicas e ferramentas de orquestração disponíveis para realização da tarefa.
\item \label{activity:develop} \textbf{Desenvolvimento: }
Aplicação das ferramentas de orquestração para criação de um ambiente propício ao ensino de computação paralela e ditribuída.
\item \label{activity:preview} \textbf{Prévia: }
Apresentação do desenvolvimento realizado até o momento.
\item \label{activity:exec} \textbf{Testes e experimentação: }
Analisar e realizar modificações necessárias para tornar o ambiente mais apropriado.


\end{enumerate}
\textbf{Cronograma}

\begin{table}[ht]
\centering
\begin{tabular}{c|ccccc}
	Etapa & Março & Abril & Maio & Junho & Julho \\ \hline
	\ref{activity:cloud} & \checkmark & & & \\
	\ref{activity:develop} & \checkmark & \checkmark & & \\
	\ref{activity:preview} & & & \checkmark & & \\
	\ref{activity:exec} & & & \checkmark & \checkmark & \checkmark

\end{tabular}
\caption{Cronograma de Atividades}

\end{table}
\section{Recursos}
Será utilizado o equipamento pessoal do orientando para o desenvolvimento. Também será utilizada uma conta gratuíta da Amazon Web Services.

\section{Resultados Esperados}
Espera-se automatizar a implantação de um ambiente cloud com todas as ferramentas necessárias para o ensino-aprendizagem de computação paralela e distribuída.

\bibliographystyle{abbrvnat}
\bibliography{ref}

\end{document}